\begin{thebibliography}{1}

\bibitem{vh}
{V.~Hordějčuk:
{\em Optimisation of Rectangular Shapes Placement by Means of Evolutionary Algorithms},
bachelor`s thesis,
2010.}

\bibitem{hynek}
{J.~Hynek:
{\em Genetické algoritmy a genetické programování},
Grada Publishing, a.s.,
2008.}

\bibitem{btreesa}
{T.C.~Chen, Y.W.~Chang:
{\em Modern Floorplanning Based on B*-Tree and Fast Simulated Annealing},
IEEE Trans. Computer-Aided Design of Integrated Circuits and Systems, Vol. 24, No. 4, pp. 637--650,
April 2006.}

\bibitem{poems}
{J.~Kubalik, J.~Faigl:
{\em Iterative Prototype Optimisation with Evolved Improvement Steps},
Lecture Notes in Computer Science, Springer,
2006.}

\bibitem{tcg}
{J.M.~Lin, Y.W.~Chang:
{\em TCG: A Transitive Closure Graph-Based Representation for Non-Slicing Floorplans},
IEEE Trans. Very Large Scale Integration System, Vol. 13, No. 2, pp. 288--292,
February 2005.}

\bibitem{tcgs}
{J.M.~Lin, Y.W.~Chang:
{\em TCG-S: Orthogonal Coupling of P*-admissible Representations for General Floorplans},
IEEE Trans. Computer-Aided Design of Integrated Circuits and Systems, Vol. 23, No. 6, pp. 968--980,
June 2004.}

\bibitem{pe2}
{M.~Lai and D.F.~Wong:
{\em Slicing Tree Is a Complete Floorplan Representation},
Proceedings of Design Automation Conference, pp. 228--232,
2001.}

\bibitem{btree}
{Y.C.~Chang, Y.W.~Chang, G.M.~Wu, S.W.~Wu:
{\em B*-Trees: A New Representation for Non-Slicing Floorplans},
Proceedings of 36th Design Automation Conference, pp. 458--463,
2000.}

\bibitem{cbl}
{X.L.~Hong, G.~Huang, T.~Cai, J.~Gu, S.~Dong, C.K.~Cheng, J.~Gu:
{\em Corner Block List: An Effective and Efficient Topological Representation of Non-Slicing Floorplan},
Proceedings of International Conference on Computer Aided Design, pp. 8--12, 
2000}

\bibitem{otree}
{P.N.~Guo, C.K.~Cheng, T.~Takahashi, T.~Yoshimura:
{\em Floorplanning Using a Tree Representation},
IEEE Transactions on Computer-Aided Design of Integrated Circuits and Systems, vol. 20, no. 2, pp. 281--289, 
February 2001.}

\bibitem{sp}
{H.~Murata, K.~Fujiyoshi, S.~Nakatake, Y.~Kajitani:
{\em VLSI Module Placement Based on Rectangle-Packing by the Sequence-Pair},
IEEE Transactions on Computer-Aided Design of Integrated Circuits and Systems, vol. 15, no. 12, pp. 1518--1524, 
1996.}

\bibitem{bsg}
{S.~Nakatake, K.~Fujiyoshi, H.~Murata, Y.~Kajitani:
{\em Module Placement on BSG-structure and IC Layout Applications},
Proceedings of International Conference on Computer Aided Design, pp. 484--491, 
1996.}

\bibitem{nphard}
{H.~Murata, K.~Fujiyoshi, S.~Nakatake, Y.~Kajitani:
{\em VLSI Module Placement Based on Rectangle-Packing by the Sequence-Pair},
1996.}

\bibitem{koza}
{J.R.~Koza:
{\em Genetic Programming. On the Programming of Computers by Means of Natural Selection},
Cambridge, MA: MIT Press,
1992.}

\bibitem{tournament}
{D.E.~Goldberg, K.~Deb:
{\em A Comparative Analysis of Selection Schemes Used in Genetic Algorithms},
Foundations of genetic algorithms, Bloomington, IN,
1991.}

\bibitem{pe}
{D.F.~Wong, C.L.~Liu:
{\em A New Algorithm for Floorplan Design},
Proceedings of Design Automation Conference, pp. 101--107,
1986.}

\bibitem{amir}
{B.S.~Baker, E.G.~Coffman, R.L.~Rivest:
{\em Orthogonal Packings in Two Dimensions},
SIAM Journal on Computing, vol. 9, no. 4, pp. 846--855,
1980.}

\bibitem{ga}
{J.H.~Holland:
{\em Adaptation in Natural and Artificial Systems},
Michigan Press, 
1975.}

\bibitem{darwin}
{C.~Darwin:
{\em O vzniku druhů přirozeným výběrem čili zachování vhodných odrůd v boji o život},
F. Klapálek, Praha, 
1914.}

\bibitem{bench}
{\tt http://vlsicad.eecs.umich.edu/BK/FPUtils/}

\bibitem{benchmcnc}
{\tt http://vlsicad.eecs.umich.edu/BK/CompaSS/results/mcnc\_opt.html}

\bibitem{benchgsrc}
{\tt http://vlsicad.eecs.umich.edu/BK/CompaSS/results/gsrc\_opt.html}

\bibitem{repository}
{\tt http://code.google.com/p/vh-master-thesis/}

\end{thebibliography}
